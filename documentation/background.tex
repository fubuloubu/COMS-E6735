Several commercially-available systems\cite{chordify,riffstation} similar in end-use exist
that use audio recognition in order to identify arrangenents of notes with moderate accuracy.
These systems are challenged by fast-pace music as slowing down the input waveform will change
the central tone present, and can have difficulty identifying individual notes when multiple
tones are present, due to the techniques utilized for this purpose. Additionally, there is no
easy way that these audio systems can accurately identify fretting positions, as there are
multiple fretting locations corresponding to the same central tone available on a guitar.
The fretting location can often be vital to correctly reproducing a piece, so this information
is important to identify, especially for beginner guitarists looking to learn.
\par
A Computer Vision application to identify fretting locations would have the benefit that controlling
the speed would not affect the ability to correctly identify musical events, and would also be
able to easily identify the notes reproduced if the tuning of the guitar were provided.
The system would find use in a amateur training scenario where it is often hard to identify the
notes being played in a specific song, or in a practice scenario where it might be useful to record
the tabulature of a session instead of having to pain-stakingly review audio or video recordings of
a session in order to identify a specific progression that was being played.
\par
There exist no commercially viable systems utilizing Computer Vision capable of accurate note
identification, either as the primary means or in concert with audio deconstruction techniques.
A review of the literature revealed several attempts at systems of similar scope and ambition to
the system described in this report\cite{GuitarTabulizer,CVFinger}. Such systems require a
particular lighting and camera setup attached to the instrument that may preclude adoption.
A more natural system that post-analyzes recorded video in a larger variety of contexts, lighting,
and different types of stringed instruments would acheive more widespread adoption.
