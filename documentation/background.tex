A review of the literature revealed unsurprisingly that a project concerning this topic is unprecedented.
Similar systems exist that typically utilize pure audio recognition to identify arrangenents of notes with 
a correctness approaching only about 85\%.
These systems can typically only recognize events such as chords, due to the difficulty identifying individual sounds from an arrangement 
\cite{chordify,riffstation}.
Additionally, these systems are challenged by fast-pace music as slowing down the input waveform will change
the central tone present (in reality, this is possible, but only by tracking the slowdown and filtering the noise).
\par

A video system to identify notes would have the benefit that controlling the speed would not affect the
ability to correctly identify musical events (within the recording speed limits of the video),
although this means the system will be unable to identify the exact tone of that note without further tonal analysis.
The system should find use in a amateur training scenario where it is often hard to identify the notes being played in a specific song, 
or in a practice scenario where it might be useful to record the tabulature of a session
instead of having to pain-stakingly review a practice session to identify a specific progression that was being played.
