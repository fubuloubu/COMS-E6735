\subsection{Location Stage}
In order to improve the detection rate for the location stage, the cascade classifier used needs to be
trained under a much greater diversity of lighting and clothing choices. This would increase the accuracy in
finding the guitar in a given frame, which would also increase the detection rate of the hands in context.
Additional investigation was performed into alternative classifiers. A classifier using pixel intensity
comparisions \cite{pico} was intended on being used as initial research indicated it had potential for
rotation-invariant object detection, essential to the application here and missing from the final
project. However, time constraints limited the implementation of the upgraded classifier.

\subsection{Guitar Modeling Stage}
Improving the fret detection framework would be a priority of this stage, as the string detection is already
sufficiently accurate enough as to be useful. Some ideas included working on line detection of much smaller
lines perpindicular to the detected strings, as the frets are underneath the strings and may therefore 
hamper line detection due to the occlusion. Other ideas may include finding intersection points that identify
a regular grid pattern on the guitar neck, which would also be useful for the identification of string-fret
pairings required for fingering positions.

\subsection{Hand Modeling Stage}
Much more work is needed for this stage as the currently utilized modeling technique is to obtain line
endpoints of an artificial "skeleton" model. This might not be the best algorithm for detection of fingertip
points, so further investiation would be required if better algorithms might produce nicer overall results.
Using the current algorithm, filtering could be employed to identify only the points that are within the region
of interest (namely the guitar neck/body using the string locations). Partial occlusion of the fingers in
many of the frames analyzed may make any algorithm to detect finger tips difficult overall however.

\subsection{Music Production Model and Output Layer}
As noted, this layer was not even attempted for the final project because it requires all of the prior
levels to working with at least a moderate level of success before reasonable results could be obtained.
Future work would focus on this layer and using motion-obtaining techniques to determine when picking
events occurred in order to match the events to the production of notes and obtain musical tablature as
the final output.
