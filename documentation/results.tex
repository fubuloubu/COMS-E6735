Using the evaluation methodology mentioned before, the accuracy of each major stage of the program was
recorded and measured for each of the five test videos, and the average per-frame accuracy was taken.
Additionally, the per-video average accuracy was then taken to show the differences between all videos.
The results are detailed in the following table (values are in percent):
\vspace{5mm}

% Table of results from prototype phase
\begin{centering}
\begin{tabular}{llll}
\toprule
 Result                   &   locate.results &   handmodel.results &   guitarmodel.results \\
\midrule
 Average Score bass-1.mp4 & 0                & 46                  & 59                    \\
 Average Score bass-2.mp4 & 0                & 23                  & 65                    \\
 Average Score bass-3.mp4 & 0                & 8                   & 64                    \\
 Average Score bass-4.mp4 & 0                & 0                   & 64                    \\
 Average Score bass-5.mp4 & 1                & 40                  & 52                    \\
 Total Average Score      & 0                & 23                  & 60                    \\
\bottomrule
\end{tabular}

\end{centering}
\vspace{5mm}

As can be seen, the overall accuracy is poor for all program layers tested.
\subsection{Location Stage}
The location stage is the worst overall, with an inability to detect the location of the guitar
in the vast majority of scenes, and a near-zero recognition rate for the location of the hands.
Overall, the location stage's accuracy is about 0.2\% over the variety of input video circumstances.
The 1\% accuracy in the 5-th video is interesting as that video was the only video in which clothing was changed.
In reviewing the results of the footage overlaid with the location footage also reveals the results
are very poor. When it does locate an object, the object is only sometimes correctly identified.
Often, the identified object was located in the background, which was white for most of the videos.
This speaks to the classifier being poorly trained.

\subsection{Guitar Modeling Stage}
The guitar modeling stage is the best overall, with a majority of frames able to find the guitar strings
with a high degree of precision. 50\% of the detection accuracy was awarded to finding exactly 4-strings
in each frame, which in most frames was easy to do. Review of the results of the footage overlaid with lines
representing the detected strings also showed the results were agreeable to human subjects.
In contrast, the frets were not found to that same degree of precision and not in the same quantity as desired.
The obtained frets were usally 3-4 in number (24 was the desired quantity) and significantly exceeded the boundary
of the guitar neck, as identified using string detection. This was verified by manual review of the overlaid
frames.
\par% From methodology
These techniques prove fairly reliable in early trials, however lighting and orientation
differences could lead to this algorithm to be somewhat fickle in identifying all of the frets.
Since the guitar should not change scale in frame too much, and the range of orientation
and lighting should be small, the idea was to continually estimate the location of each feature
(string/fret) until a model is built that fully matches the expected number of strings and frets.
This fully realized model could then be tracked for scale and orientation changes in order to
accurately track where the guitar is moving during playing.
The end product could leverage this by instructing a live user to move around their hands until
the guitar model layer has successfully accomplished this, and inform the user when it is unable
to track these quantities during operation.

\subsection{Hand Modeling Stage}
The hand modeling stage was met with moderate initial success. Fingertip points were identified in almost
half the frames in two of the five videos. However, the framing used for skin detection of both hands
were not able to leverage the hand locations from the Location Stage, so the locations were set manually.
This artificially deflated the results of this layer. Upon review of the overlaid results, the identified
fingertip points were often poorly selected. The algorithm used identified line ends of an artificially
produced "skeleton", which could be an underlying source of error as the algorithm may not be robust enough.

\subsection{Music Production Model and Output Layer}
This layer was not attempted. No results were obtained.
